\documentclass{article}
\usepackage{listings}
\usepackage{booktabs}
\usepackage{here}
\usepackage{subcaption}
\usepackage{tikz}
\usepackage{amsmath}
\usepackage{graphicx}
\usepackage{hyperref}


\newcommand{\code}[1]{{\tt #1}}

% decreas margin size
\usepackage[margin=0.7in]{geometry}

\newcommand\blankpage{%
	\null
	\thispagestyle{empty}%
	\addtocounter{page}{-1}%
	\newpage}

\title{
	\normalfont\normalsize
	\textsc{Data Analytics (2024-25)\\
	Universit\`a della Svizzera italiana}\\
	% \vspace{2pt}
	\rule{\linewidth}{0.5pt}\\
	% \vspace{5pt}
	{\huge College network\\
	\small Course Assignment N.13}\\
	% \vspace{5pt}
	\rule{\linewidth}{1pt}\\
	\vspace{5pt}
}

\author{
	Paolo Deidda (\text{paolo.deidda@usi.ch}) \\ 
	Andrea Luca Perugini (\text{andrea.perugini@usi.ch})\\
	\url{https://github.com/USI-Projects-Collection/DA-College_network.git}
	}


\begin{document}
\maketitle

\tableofcontents

\vspace{50pt}
% \rule{\linewidth}{0.5pt}

\vspace*{\fill}

\subsection*{Setup}

To run the code and reproduce the figures or outputs, you can either run the Jupyter Notebook directly on Google Colab, or follow the setup and execution instructions provided in the \texttt{README.md} file included in the repository.

\section*{Introduction}

This document presents an analysis of a dataset of private messages exchanged on a UC Irvine social network. The dataset includes 1,899 users and 59,835 messages, each represented as a directed interaction from a sender (SRC) to a recipient (TGT) with a Unix timestamp (UNIXTS).

The report is structured into two main parts. The first part focuses on data exploration and temporal activity analysis, where we study messaging behavior over time and among different types of users. The second part covers network-based analysis, where we investigate the structure of the social graph, user centrality, interaction patterns, and community structures
\newpage

\section{Data Analysis}

\subsection{Setup and Data Loading}


\subsection{Data Exploration}
\subsubsection{Basic Statistics}
\subsubsection{Message Frequency Over Time}

\subsection{Network Construction}
% \subsubsection{Creating the Graph}
\subsubsection{Degree Analysis}
\subsubsection{Identifying Isolated and One-Way Nodes}

\subsection{User Analysis}
\subsubsection{Top Senders and Receivers}
\subsubsection{PageRank Centrality}
\subsubsection{Edge Reciprocity}

\subsection{Temporal Analysis}
\subsubsection{Messages by Weekday}
\subsubsection{Top User Activity Analysis}



\newpage

\section{Netwrok Analysis}
\subsection{Network Structure Analysis}
\label{sec:network_structure}

The network architecture exhibits key social network characteristics. The calculated \textbf{graph density} is \texttt{0.0056}, confirming a sparse structure typical for such networks where most potential connections are absent. Despite sparsity, analysis of \textbf{weakly connected components (WCCs)} reveals high overall connectivity, with a single giant component encompassing 1893 nodes (over 99.6\% of users), facilitating potential information diffusion.

Considering message directionality (Figure~\ref{fig:scc_visualization}), the \textbf{strongly connected components (SCCs)} analysis identified 601 components, dominated by a large core of 1294 users ($\approx$ 68\%) capable of reciprocal communication. The numerous smaller components highlight a significant periphery, indicating a distinct core-periphery structure critical to information flow dynamics.

Local structure analysis shows moderate \textbf{unweighted clustering} (\texttt{0.087}), suggesting inherent social grouping beyond random chance. However, the \textbf{weighted clustering coefficient} (using message counts) is markedly lower (\texttt{0.0018}). This significant difference implies that while structural triads are present, the intensity of communication within these local groups is often unevenly distributed.


\begin{figure}[h!]
    \centering
    \includegraphics[width=0.9\textwidth]{../scc.png}
    \caption{Network visualization highlighting Strongly Connected Components (SCCs). Nodes coloured by SCC category (Red: Core; Grey: Peripheral; Others: Multi-Node SCCs). Size reflects degree.}
    \label{fig:scc_visualization}
\end{figure}

\subsection{Expanded Centrality Analysis}
\label{sec:centrality_analysis}

Beyond degree and PageRank (Section 1.4), \textbf{Eigenvector Centrality} and \textbf{Betweenness Centrality} provide further insights into node influence and structural roles.

Eigenvector centrality highlights influence via connections to other influential nodes. Top users like \texttt{User 1624} (score $\approx$ 0.47) and \texttt{User 398} ($\approx$ 0.30) demonstrate high influence despite not having top degrees, indicating strategic positioning. Others, like \texttt{User 105} ($\approx$ 0.27), combine high Eigenvector scores with high out-degree and prominence in other rankings.

Betweenness centrality identifies crucial network brokers. Top users \texttt{User 323} ($\approx$ 0.12), \texttt{User 1624} ($\approx$ 0.08), and \texttt{User 105} ($\approx$ 0.08) score highly, reinforcing their central and bridging roles suggested by other measures. The appearance of other high-degree/PageRank users (like \texttt{User 32}, \texttt{User 103}) in the top 10 for Betweenness indicates that popular users can also be important intermediaries.

Analysis of \textbf{Spearman rank correlations} reveals strong positive associations between all centrality measures (most $\rho > 0.7$). In-degree shows particularly high correlation with PageRank ($\rho \approx 0.93$) and Eigenvector ($\rho \approx 0.90$). Betweenness centrality displays slightly weaker, yet still strong, correlations ($\rho \approx 0.72 - 0.81$), suggesting it captures the most distinct structural aspect (brokerage). Overall, these strong correlations indicate that different facets of importance (popularity, influence via connections, brokerage) are significantly intertwined in this network.
\subsection{Network Patterns: Degree Correlation and Activity Symmetry}
\label{sec:degree_patterns}

Examining network mixing patterns, the \textbf{degree assortativity coefficient} was \texttt{-0.1375}. This negative value indicates \textbf{disassortativity}: high-degree users tend to connect with low-degree users, suggesting a hub-and-spoke structure rather than interconnected cores of highly active users.

In contrast to this network-level pattern, individual user activity shows symmetry. A strong positive \textbf{Pearson correlation ($\rho \approx 0.83$)} exists between user in-degree and out-degree, meaning users receiving many messages typically also send many. Thus, while the network structure promotes connections between different activity levels, highly engaged individuals tend to be active communicators in both directions.


% \begin{figure}[h!]
%     \centering
%     \includegraphics[width=0.7\textwidth]{../inoutdegree.png}
%     \caption{Scatter plot of user in-degree vs. out-degree. The strong positive correlation ($\rho \approx 0.83$) indicates symmetry in individual user activity.}
%     \label{fig:in_out_corr}
% \end{figure}

\subsection{Community Structure}
\label{sec:community_structure}

To uncover meso-scale organization, we applied the \textbf{Louvain algorithm} for community detection. The analysis partitioned the network into \textbf{16 distinct communities} with a \textbf{modularity score} of approximately \texttt{0.3607}. This modularity value indicates a meaningful, albeit imperfect, community structure significantly better than random partitioning.

The detected communities exhibit considerable heterogeneity. Sizes range from large groups (e.g., Community 9: 361 nodes, Community 2: 305 nodes) down to three 2-node communities (likely isolated pairs). Internal structure also varies; the largest communities tend to be internally sparse, while some smaller communities show higher internal density or higher average internal centrality metrics. This community structure is visualized in Figure~\ref{fig:community_visualization}, illustrating the arrangement and relative sizes of the main groups.


\begin{figure}[h!]
    \centering
    \includegraphics[width=0.9\textwidth]{../communities.png}
    \caption{Network visualization with nodes coloured by detected community membership (Louvain algorithm).}
    \label{fig:community_visualization}
\end{figure}

The presence of this discernible community structure provides valuable insights into the social segmentation within the network.

\end{document}